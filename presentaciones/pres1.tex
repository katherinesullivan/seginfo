\documentclass{beamer}
\usetheme{Madrid}

%\useoutertheme{miniframes}
%\useoutertheme{infolines}
%\useoutertheme{smoothbars}
%\useoutertheme{sidebar}
\useoutertheme{split}
%\useoutertheme{shadow}
%\useoutertheme{tree}
%\useoutertheme{smoothtree}
\usepackage[utf8]{inputenc}

\title{A formal Analysis for Capturing Replay Attacks in Cryptographic Protocols}
\subtitle{}
\author{Han Gao, Chiara Bodei, Pierpaolo Degano, y Hanne Riis Nielson}
\institute{Katherine Sullivan \newline FCEIA - UNR}
\date{}

\begin{document}

\begin{frame}
    \titlepage
\end{frame}

\AtBeginSection[]
  {
     \begin{frame}<beamer>
     \frametitle{Índice}
     \tableofcontents[currentsection]
     \end{frame}
  }

\begin{frame}
    \frametitle{Índice}
    \tableofcontents
\end{frame}

\section{Introducción}
\begin{frame}
    \frametitle{Introducción}
    \pause
    \begin{itemize}
        \item<2-> ¿En qué consisten los ataques por repetición?
            \begin{itemize}
                \item<3-> Tipo de ataque en el que un adversario intercepta y retransmite datos previamente capturados para intentar ganar acceso no autorizado o causar un mal funcionamiento en un sistema.
            \end{itemize}
        \vspace{0.75cm}
        \item<4-> ¿Cómo se hará el análisis formal para capturar ataques por repetición?
            \begin{itemize}
                \item<5-> A través de la extensión de L\textsc{y}S\textsc{a}, un álgebra de procesos, y su respectivo análisis de flujo de control, con anotaciones de sesiones.
            \end{itemize}
    \end{itemize}
\end{frame}

\section{Cálculo L\textsc{y}S\textsc{a}}
\subsection{¿Qué es L\textsc{y}S\textsc{a}?}

\begin{frame}
    \frametitle{¿Qué es L\textsc{y}S\textsc{a}?}
    \pause
    Es un álgebra de procesos desarrollada en Automatic Validation of Protocol Narration (2003) y en Static Validation of Security Protocols (2005) por Chiara Bodei, Mikael Buchholtz, Pierpaolo Degano, Flemming Nielson y Hanne Riis Nielson con ciertas particularidades:
    \pause[2]
    \begin{itemize}
        \item<3-> No hay canales: en L\textsc{y}S\textsc{a} todos los procesos tienen acceso solo a un único canal de comunicación global. 
        \item<4-> Las verificaciones asociadas con \textit{inputs} (recepciones de mensajes) y desencriptaciones son expresadas usando \textit{pattern matching}.

    \end{itemize}

\end{frame}

\subsection{Sintaxis}

\begin{frame}
    \frametitle{Sintaxis L\textsc{y}S\textsc{A}}
    \pause
    La sintaxis de expresiones resulta simple de comprender, estando conformada por nombres, variables y expresiones encriptadas. Vale la pena detenerse en la sintaxis de procesos.
    \pause[3]
    \begin{align*}
    E &::= n | x | \{ E_1, ..., E_k \}_{E_0}\\
    P &::= \begin{aligned}[t]
            &\langle E_1, \dots, E_k \rangle.P \hspace{0.2cm}
            &&\text{(envío de msj)} \\
            &\mid (E_1, \ldots, E_j; x_{j+1}, \ldots, x_k).P \hspace{0.1cm} &&\text{(recepción de msj)} \\
            &\mid \text{decrypt } E \text{ as } \{E_1, \ldots, E_j; x_{j+1}, \ldots, x_k\}^{l}_{E_0} \text{ in } P \hspace{0.2cm} &&\text{(desencriptación)} \\
            &\mid (\nu n)P \hspace{0.2cm} &&\text{(nuevo nombre)}\\ 
            &\mid P_1 \mid P_2 \hspace{0.2cm} &&\text{(paralelismo)}\\
            &\mid !P \hspace{0.2cm} &&\text{(replicación)}\\ 
            &\mid 0 \hspace{0.2cm} &&\text{(proceso nulo)}\\
        \end{aligned}
\end{align*}
\end{frame}

\begin{frame}
    \frametitle{Sintaxis L\textsc{y}S\textsc{A} extendida I}
    \pause
    Ahora cada término y proceso llevará un identificador de la sesión a la que pertenece. 
    \pause[3]
    \begin{center}
    \includegraphics[scale=0.4]{sintaxisextendida.png}
    \end{center}
    \pause[4]
    Pero, ¿cómo se mapean términos y procesos estándar a unos de la sintaxis extendida? \pause[5]Añadiendo identificadores de sesión inductivamente a través de dos funciones: $\mathcal{F}$ y $\mathcal{T}$.
\end{frame}

\begin{frame}
    \frametitle{Sintaxis L\textsc{y}S\textsc{A} extendida II}
    \pause
    \begin{center}
    \includegraphics[scale=0.55]{fcssessionids.png}
    \end{center}
\end{frame}

\subsection{Semántica operacional}

\begin{frame}
    \frametitle{Semántica operacional I}
    \pause
    Se consideran dos variantes de la relación de reducción $\rightarrow_\mathcal{R}$, identificadas por una diferente instanciación de la relación $\matchcal{R}$, que decora la relación de transición. 
    \pause[3]
    \vspace{0.7cm}
    
    Una variante ($\rightarrow_{RM}$) aprovecha las anotaciones, la otra ($\rightarrow$) las descarta: esencialmente, la primera semántica verifica la frescura de los mensajes, mientras que la otra no lo hace.
\end{frame}

\begin{frame}
    \frametitle{Semántica operacional II}
    \pause
    Antes de pasar a la definición de la relación necesitamos de dos definiciones:
    \pause[3]
    \begin{itemize}
        \item<4-> La relación de equivalencia $V_1 \overset{f}{=} V_2$ definida como la menor equivalencia sobre $Val$ que (de manera inductiva) ignora los identificadores de sesión.
        \item<5-> La función $\mathcal{I}: Val \rightarrow SID$ de extracción de identificadores de sesión definida como sigue:
            $$\mathcal{I}([n]_s) = s$$
            $$\mathcal{I}([{v_1, \dots, v_k}_{v_0}]_s) = s$$
    \end{itemize}
\end{frame}

\begin{frame}
    \frametitle{Semántica operacional III}
    \pause
    Ahora sí, pasemos a la definición de la relación de reducción.
    \pause[3]
    \begin{center}
        \includegraphics[scale=0.55]{relaciondereduccion.png}
    \end{center}
\end{frame}

\begin{frame}
    \frametitle{Semántica operacional IV}
    \pause
    Con la relación de reducción definida podemos pasar a dar una de las definiciones más relevantes: la de frescura.
    \pause[3]
    \vspace{0.7cm}
    
    \textbf{Def. Frescura.} Un proceso $P$ asegura la propiedad de frescura si, para todas las ejecuciones posibles $P \rightarrow^*_{\mathcal{R}} P' \rightarrow P''$ cuando $P' \rightarrow P''$ se deriva usando $(Dec)$ en
\[
\text{{decrypt }} [\{ V_1, \ldots, V_k\}_{V_0} ]_s \text{{ as }} \left\{ V_1', \ldots, V_j'; x_{j+1}, \ldots, x_k \right\}^l_{V'} \text{{ in }} P,
\]
existe al menos un $i$ $(1 \leq i \leq j)$ tal que $\mathcal{I}(V_i) = \mathcal{I}(V_i')$.

\end{frame}

\begin{frame}
    \frametitle{Ejemplo: Protocolo Wide Mouthed Frog}
    \pause
    Se usa una versión simplificada (sin timestamps) del protocolo WMF, un protocolo de gestión de claves simétrico cuyo objetivo es establecer un secreto
    clave de sesión $K_{ab}$ entre A y B que comparten sus claves secretas
    $K_A$ y $K_B$ con un servidor de confianza S, para mostrar como ejemplo.
    \pause[3]
    
    \vspace{0.5cm}
    Narración:
    \begin{center}
        \includegraphics[scale=0.5]{narracionWMF.png}
    \end{center}
\end{frame}

\begin{frame}
    \frametitle{Ejemplo: protocolo Wide Mouthed Frog}
    \pause
    Especificación L\textsc{y}S\textsc{a}:
    \pause[3]
    \begin{center}
        \includegraphics[scale=0.5]{espWMF.png}
    \end{center}
\end{frame}

\subsection{Análisis estático}

\begin{frame}
    \frametitle{Análisis de términos I}
    \pause
    \begin{itemize}

        \item<2-> $\kappa \subseteq \wp (\text{Val}^*)$ es el entorno de red abstracto que incluye todas las tuplas que forman un mensaje que puede fluir en la red.

        \item<3-> $\rho : X \rightarrow \wp (\text{Val})$ es el entorno de variables que asigna las variables a los conjuntos de valores a los que pueden estar vinculadas.
        
        \item<4-> Se utilizará $\rho \vdash \mathcal{E} : \vartheta$ para indicar que el conjunto $\vartheta$ es una estimación aceptable (una sobreaproximación correcta) de los posibles valores a los que el término $\mathcal{E}$ puede evaluar en el entorno $\rho$. 
        
        \item<5->Se emplean dos tipos de pruebas de pertenencia: $V \in \vartheta$ para comprobar si $V$ está en el conjunto $\vartheta$ y $V \propto \vartheta$ para probar si hay un valor $V'$ en $\vartheta$ que es igual a $V$, ignorando las anotaciones.
 
    \end{itemize}
\end{frame}

\begin{frame}
    \frametitle{Análisis de términos II}
    \pause
    \begin{center}
        \includegraphics[scale=0.6]{analisisterminos.png}
    \end{center}
\end{frame}

\begin{frame}
    \frametitle{Análisis de procesos I}
    \pause
    \begin{itemize}
        \item<2-> $\psi$ es un conjunto posiblemente vacío de componentes de error que recopila una sobreaproximación de violaciones de frescura. Un $l \in \psi$ significa que el valor vinculado después de un descifrado exitoso, marcado con la etiqueta $l$, viola las anotaciones de frescura y, por lo tanto, no está permitido.

        \item<3-> Se utiliza el símbolo $\rho, \kappa \vDash_{\text{RM}} \mathcal{P} : \psi$ para expresar que $\rho$, $\kappa$, y $\psi$ son estimaciones de análisis válidas para el proceso $\mathcal{P}$. 
        
    \end{itemize}
\end{frame}

\begin{frame}
    \frametitle{Análisis de procesos II}
    \pause
    \begin{center}
        \includegraphics[scale=0.6]{analisisprocesos.png}
    \end{center}
\end{frame}

\begin{frame}
    \frametitle{Análisis de procesos III}
    \pause
    \begin{center}
        \includegraphics[scale=0.6]{analisisprocesos2.png}
    \end{center}
\end{frame}

\subsection{Propiedades}

\begin{frame}
    \frametitle{Propiedades I}
    \pause
    Las estimaciones son resistentes a la sustitución de términos cerrados por variables.
    \pause[3]
    \begin{center}
        \includegraphics[scale=0.65]{lema1.png}
    \end{center}
\end{frame}

\begin{frame}
    \frametitle{Propiedades II}
    \pause
    Una estimación para un proceso extendido
    $P$ es válida para cualquier proceso extendido congruente con $P$.
    \pause[3]
    \begin{center}
        \includegraphics[scale=0.65]{lema2.png}
    \end{center}
\end{frame}

\begin{frame}
    \frametitle{Propiedades III}
    \pause
    El resultado del análisis para un proceso es válido para sus derivados.
    \pause[3]
    \begin{center}
        \includegraphics[scale=0.65]{teorema1.png}
    \end{center}
\end{frame}

\begin{frame}
    \frametitle{Propiedades IV}
    \pause
       Si el conjunto de etiquetas 
        $\psi$ es vacío, entonces el monitor de referencia no puede abortar el proceso $\mathcal{P}$, ie. $$\nexists \mathcal{Q, Q'} / \mathcal{P} \rightarrow^*_{\mathcal{R}} \mathcal{Q} \rightarrow_{\text{RM}} \mathcal{Q'} \wedge \mathcal{P} \rightarrow^*_{\mathcal{R}} \mathcal{Q} \nrightarrow_{\text{RM}}$$
    \pause[3]
    \begin{center}
        \includegraphics[scale=0.55]{teorema2.png}
    \end{center}
\end{frame}

\begin{frame}
    \frametitle{Ejemplo: análisis estático de WMF}
    \pause
    Análisis:
    \pause[3]
    \begin{center}
        \includegraphics[scale=0.5]{analisiswmf.png}
    \end{center}
\end{frame}

\begin{frame}
    \frametitle{Ejemplo: análisis estático de WMF}
    \pause
    Posible ataque:
    \pause[3]
    \begin{center}
        \includegraphics[scale=0.65]{posibleataquewmf.png}
    \end{center}
\end{frame}

\subsection{Modelado de atacantes}

\begin{frame}
    \frametitle{Modelado de atacantes I}
    \pause
    \begin{itemize}
    \item<2-> Se dice que un proceso \(\mathcal{P}_{\text{sys}}\) tiene el tipo \((\mathcal{N}_f, \mathcal{A}_\kappa, \mathcal{A}_{\text{Enc}})\) cuando:
    \begin{enumerate}
        \item Es cerrado.
        \item Todos los nombres libres de \(\mathcal{P}_{\text{sys}}\) están en \(\mathcal{N}_f\).
        \item Todas las aridades utilizadas para enviar o recibir están en \(\mathcal{A}_\kappa\).
        \item Todas las aridades utilizadas para encriptar o desencriptar están en \(\mathcal{A}_{\text{Enc}}\).
    \end{enumerate}
    \item<3-> Se considerarán atacantes Dolev-Yao activos.
    \end{itemize}
\end{frame}

\begin{frame}
    \frametitle{Modelando atacantes II}
    \pause
    Atacantes Dolev-Yao activos:
    \pause[3]
    \begin{center}
        \includegraphics[scale=0.5]{atacantesdolevyao.png}
    \end{center}
\end{frame}

\begin{frame}
    \frametitle{Modelando atacantes III}
    \pause
    Se define una fórmula \(\mathcal{F}^{\text{DY}}_{\text{RM}}\) del tipo \((\mathcal{N}_f, \mathcal{A}_\kappa, \mathcal{A}_{\text{Enc}})\) como la conjunción de los cinco componentes en la tabla mostrada anteriormente
    \pause[3]
    
    y se establece que la fórmula \(\mathcal{F}^{\text{DY}}_{\text{RM}}\) es capaz de caracterizar el efecto potencial de todos los atacantes \(\mathcal{Q}\) del tipo \((\mathcal{N}_f, \mathcal{A}_\kappa, \mathcal{A}_{\text{Enc}})\).
    \pause[4]
    \begin{center}
        \includegraphics[scale=0.55]{dolevyao.png}
    \end{center}
\end{frame}

\section{Resultados principales}

\begin{frame}
    \frametitle{Frescura dinámica}
    \pause
    Se dice que \(\mathcal{P}_{\text{sys}}\) garantiza frescura dinámica con respecto a las anotaciones en \(\mathcal{P}_{\text{sys}}\) si el monitor de referencia \(\mathcal{RM}\) no puede abortar \(\mathcal{P}_{\text{sys}} \,|\, \mathcal{Q}\) independientemente de la elección del atacante \(\mathcal{Q}\).
    \pause[3]

    Se muestra que frescura estática implica frescura dinámica.
    \pause[4]
    \begin{center}
        \includegraphics[scale=0.55]{frescuradinamica.png}
    \end{center}
\end{frame}

\begin{frame}
    \frametitle{Implementación}
    \pause
    Para obtener una implementación, se transforma el análisis en una formación lógicamente equivalente escrita en Alternation-free Least Fixed Point logic (ALFP) (Nielson-Seidl-Nielson, 2002), y se utiliza el Succinct Solver (Nielson-Seidl-Nielson, 2002), que calcula la interpretación mínima de los símbolos predicados en una fórmula ALFP dada.
\end{frame}

\begin{frame}
    \frametitle{Validación del protocolo de Needham-Schroeder I}
    \pause
    \begin{center}
        \includegraphics[scale=0.6]{needhamschroeder.png}
    \end{center}
\end{frame}

\begin{frame}
    \frametitle{Validación del protocolo de Needham-Schroeder II}
    \pause
    La solución propuesta por Needham y Schroeder implica la introducción de un nuevo fresco llamado $N'_a$. Después de la corrección, el protocolo incluirá una solicitud adicional del nuevo valor $N'_a$ por parte de A a B, y este valor se enviará al servidor para su retorno encriptado.

    En el nuevo protocolo corregido, las primeras tres etapas del intercambio de claves se modifican como sigue:

    \begin{center}
        \includegraphics[scale=0.6]{needhamschroeder2.png}
    \end{center}

    Después de aplicar el análisis al protocolo corregido, el resultado indica que no hay violaciones posibles, es decir, $\psi = \emptyset$. 
\end{frame}

\section{Comentarios finales}

\begin{frame}
    \frametitle{Comentarios finales}
    \pause 
    \begin{itemize}
    \item<2-> Trabajos relacionados: Authenticity by Tagging and Typing (Bugliesi-Focardi-Maffei, 2004) - Types and Effects for Asymmetric Cryptographic Protocols (Gordon-Jeffrey, 2002).
    \item<3-> Este trabajo está enmarcado en un proyecto donde varias propiedades de comunicación de protocolos son analizadas mediante anotaciones y fácilmente se puede combinar con otro tipo de anotaciones, por ejemplo anotaciones de confidencialidad o anotaciones para el abordaje de type flaw attacks.
    \end{itemize}
\end{frame}

\end{document}

