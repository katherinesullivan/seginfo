\documentclass{beamer}
\usetheme{Madrid}

%\useoutertheme{miniframes}
%\useoutertheme{infolines}
%\useoutertheme{smoothbars}
%\useoutertheme{sidebar}
\useoutertheme{split}
%\useoutertheme{shadow}
%\useoutertheme{tree}
%\useoutertheme{smoothtree}
\usepackage[utf8]{inputenc}

\title{A formal Analysis for Capturing Replay Attacks in Cryptographic Protocols}
\subtitle{}
\author{Han Gao, Chiara Bodei, Pierpaolo Degano, y Hanne Riis Nielson}
\institute{Katherine Sullivan \newline FCEIA - UNR}
\date{}

\begin{document}

\begin{frame}
    \titlepage
\end{frame}

\AtBeginSection[]
  {
     \begin{frame}<beamer>
     \frametitle{Índice}
     \tableofcontents[currentsection]
     \end{frame}
  }

\begin{frame}
    \frametitle{Índice}
    \tableofcontents
\end{frame}

\section{Introducción}
\begin{frame}
    \frametitle{Introducción}
    \pause
    \begin{itemize}
        \item<2-> ¿En qué consisten los ataques por repetición?
            \begin{itemize}
                \item<3-> Tipo de ataque en el que un adversario intercepta y retransmite datos previamente capturados para intentar ganar acceso no autorizado o causar un mal funcionamiento en un sistema
            \end{itemize}
        \vspace{0.75cm}
        \item<4-> ¿Cómo se hará el análisis formal para capturar ataques por repetición?
            \begin{itemize}
                \item<5-> A través de la extensión de L\textsc{y}S\textsc{A}, un álgebra de procesos, con anotaciones de sesiones, y la extensión de su respectivo análisis \newline de flujo de control.
            \end{itemize}
    \end{itemize}
\end{frame}

\section{Cálculo L\textsc{y}S\textsc{A}}
\subsection{¿Qué es L\textsc{y}S\textsc{A}?}

\begin{frame}
    \frametitle{¿Qué es L\textsc{y}S\textsc{A}?}
    \pause
    Es un álgebra de procesos desarrollada en Automatic Validation of Protocol Narration(2003) y en Static Validation of Security Protocols (2005) por Chiara Bodei, Mikael Buchholtz, Pierpaolo Degano, Flemming Nielson y Hanne Riis Nielson con ciertas particularidades:
    \pause[2]
    \begin{itemize}
        \item<3-> No hay canales: en L\textsc{y}S\textsc{A} todos los procesos tienen acceso solo a un único canal de comunicación global. 
        \item<4-> verificaciones asociadas con \textit{inputs} (recepciones de mensajes) y desencriptaciones son expresadas usando \textit{pattern matching}.

    \end{itemize}

\end{frame}

\subsection{Sintaxis}

\begin{frame}
    \frametitle{Sintaxis L\textsc{y}S\textsc{A}}
    \pause
    La sintaxis de expresiones resulta simple de comprender, estando conformada por nombres, variables y expresiones encriptadas. Vale la pena detenerse en la sintaxis de procesos.
    \pause[3]
    \begin{align*}
    E &::= n | x | \{ E_1, ..., E_k \}_{E_0}\\
    P &::= \begin{aligned}[t]
            &\langle E_1, \dots, E_k \rangle.P \hspace{0.2cm}
            &&\text{(envío de msj)} \\
            &\mid (E_1, \ldots, E_j; x_{j+1}, \ldots, x_k).P \hspace{0.1cm} &&\text{(recepción de msj)} \\
            &\mid \text{decrypt } E \text{ as } \{E_1, \ldots, E_j; x_{j+1}, \ldots, x_k\}^{l}_{E_0} \text{ in } P \hspace{0.2cm} &&\text{(desencriptación)} \\
            &\mid (\nu n)P \hspace{0.2cm} &&\text{(nuevo nombre)}\\ 
            &\mid P_1 \mid P_2 \hspace{0.2cm} &&\text{(paralelismo)}\\
            &\mid !P \hspace{0.2cm} &&\text{(replicación)}\\ 
            &\mid 0 \hspace{0.2cm} &&\text{(proceso nulo)}\\
        \end{aligned}
\end{align*}
\end{frame}

\begin{frame}
    \frametitle{Sintaxis L\textsc{y}S\textsc{A} extendida I}
    \pause
    Ahora cada término y proceso llevará un identificador de la sesión a la que pertenece. 
    \pause[2]
    \begin{center}
    \includegraphics[scale=0.4]{sintaxisextendida.png}
    \end{center}
    \pause[3]
    Pero, ¿cómo mapeamos términos y procesos estándar a unos de la sintaxis extendida? \pause[4] Añadiendo identificadores de sesión inductivamente a través de dos funciones: $\mathcal{F}$ y $\mathcal{T}$.
\end{frame}

\begin{frame}
    \frametitle{Sintaxis L\textsc{y}S\textsc{A} extendida II}
    \pause
    \begin{center}
    \includegraphics[scale=0.55]{fcssessionids.png}
    \end{center}
\end{frame}

\subsection{Semántica operacional}

\begin{frame}
    \frametitle{Semántica operacional I}
    \pause
    Consideramos dos variantes de la relación de reducción $\rightarrow_\mathcal{R}$, identificadas por una diferente instanciación de la relación $\matchcal{R}$, que decora la relación de transición. 
    \pause[2]
    \vspace{1cm}
    
    Una variante ($\rightarrow_{RM}$) aprovecha las anotaciones, la otra ($\rightarrow$) las descarta: esencialmente, la primera semántica verifica la frescura de los mensajes, mientras que la otra no lo hace.
\end{frame}

\begin{frame}
    \frametitle{Semántica operacional II}
    \pause
    Antes de pasar a la definición de la relación necesitamos de dos definiciones:
    \pause[2]
    \begin{itemize}
        \item La relación de equivalencia $V_1 \overset{f}{=} V_2$ definida como la menor equivalencia sobre $Val$ que (de manera inductiva) ignora los identificadores de sesión.
        \item La función $\mathcal{I}: Val \rightarrow SID$ de extracción de identificadores de sesión definida como sigue:
            $$\mathcal{I}([n]_s) = s$$
            $$\mathcal{I}([{v_1, \dots, v_k}_{v_0}]_s) = s$$
    \end{itemize}
\end{frame}

\begin{frame}
    \frametitle{Semántica operacional III}
    \pause
    Ahora sí, pasemos a la definición de la relación de reducción.
    \pause[2]
    \begin{center}
        \includegraphics[scale=0.55]{relaciondereduccion.png}
    \end{center}
\end{frame}

\begin{frame}
    \frametitle{Semántica operacional IV}
    \pause
    Con la relación de reducción definida podemos pasar a dar una de las definiciones más relevantes: la de frescura.
    \pause[2]
    \vspace{0.7cm}
    
    \textbf{Def. Frescura.} Un proceso $P$ asegura la propiedad de frescura si, para todas las ejecuciones posibles $P \rightarrow^*_{\mathcal{R}} P' \rightarrow P''$ cuando $P' \rightarrow P''$ se deriva usando $(Dec)$ en
\[
\text{{decrypt }} [\{ V_1, \ldots, V_k\}_{V_0} ]_s \text{{ as }} \left\{ V_1', \ldots, V_j'; x_{j+1}, \ldots, x_k \right\}^l_{V'} \text{{ in }} P,
\]
existe al menos un $i$ $(1 \leq i \leq j)$ tal que $\mathcal{I}(V_i) = \mathcal{I}(V_i')$.

\end{frame}

\begin{frame}
    \frametitle{Ejemplo: protocolo Wide Mouthed Frog}
    \pause
    Se usa una versión simplificada (sin timestamps) del protocolo WMF, un protocolo de gestión de claves simétrico cuyo objetivo es establecer un secreto
    clave de sesión $K_{ab}$ entre A y B que comparten sus claves secretas
    $K_A$ y $K_B$ , respectivamente, con un servidor de confianza S.
    \pause[2]
    
    \vspace{0.7cm}
    Narración:
    \begin{center}
        \includegraphics[scale=0.5]{narracionWMF.png}
    \end{center}
\end{frame}

\begin{frame}
    \frametitle{Ejemplo: protocolo Wide Mouthed Frog}
    \pause
    Especificación L\textsc{y}S\textsc{a}:
    \pause[2]
    \begin{center}
        \includegraphics[scale=0.5]{espWMF.png}
    \end{center}
\end{frame}


\section{Análisis estático}

\subsection{Definiciones}

\begin{frame}
    \frametitle{Definiciones}
    \pause
    Especificación L\textsc{y}S\textsc{a}:
    \pause[2]
    \begin{center}
        \includegraphics[scale=0.5]{espWMF.png}
    \end{center}
\end{frame}

\subsection{Análisis de términos y procesos}

\begin{frame}
    \frametitle{Análisis de términos y procesos}
    \pause
    Especificación L\textsc{y}S\textsc{a}:
    \pause[2]
    \begin{center}
        \includegraphics[scale=0.5]{espWMF.png}
    \end{center}
\end{frame}

\subsection{Propiedades Semánticas}
\begin{frame}
    \frametitle{Propiedades semánticas}
    \pause
    Especificación L\textsc{y}S\textsc{a}:
    \pause[2]
    \begin{center}
        \includegraphics[scale=0.5]{espWMF.png}
    \end{center}
\end{frame}

\section{Modelado de atacantes}

\begin{frame}
    \frametitle{Modelado de atacantes}
    \pause
    Especificación L\textsc{y}S\textsc{a}:
    \pause[2]
    \begin{center}
        \includegraphics[scale=0.5]{espWMF.png}
    \end{center}
\end{frame}

\section{Resultados principales}

\begin{frame}
    \frametitle{Frescura dinámica}
    \pause
    Especificación L\textsc{y}S\textsc{a}:
    \pause[2]
    \begin{center}
        \includegraphics[scale=0.5]{espWMF.png}
    \end{center}
\end{frame}

\begin{frame}
    \frametitle{Implementación y complejidad}
    \pause
    Especificación L\textsc{y}S\textsc{a}:
    \pause[2]
    \begin{center}
        \includegraphics[scale=0.5]{espWMF.png}
    \end{center}
\end{frame}

\begin{frame}
    \frametitle{Validación del protocolo de Needham-Schroeder}
    \pause
    Especificación L\textsc{y}S\textsc{a}:
    \pause[2]
    \begin{center}
        \includegraphics[scale=0.5]{espWMF.png}
    \end{center}
\end{frame}

\section{Comentarios finales}

\begin{frame}
    \frametitle{Comentarios finales}
    \pause 
    \begin{itemize}
    \item<2-> Trabajo relacionado: .
    \item<3-> Está enmarcado en ... .
    \end{itemize}
\end{frame}

\end{document}

