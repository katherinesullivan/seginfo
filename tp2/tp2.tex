\documentclass[11pt]{article}
\usepackage[utf8]{inputenc}
\usepackage{amsfonts}
\usepackage{natbib}
\usepackage{graphicx}
\usepackage{amsmath}
\usepackage{amssymb}
\usepackage{mathrsfs} % Cursive font
\usepackage{graphicx}
\usepackage{ragged2e}
\usepackage{fancyhdr}
\usepackage{nameref}
\usepackage{wrapfig}
\usepackage{listings}

% to set spacing between lines
\usepackage{setspace}

% to rotate content 90 degrees
\usepackage{lscape}



% defining command for the curly arrow
\newcommand{\curly}{\mathrel{\leadsto}}


\usepackage{mathtools}
\usepackage{xparse} \DeclarePairedDelimiterX{\Iintv}[1]{\llbracket}{\rrbracket}{\iintvargs{#1}}
\NewDocumentCommand{\iintvargs}{>{\SplitArgument{1}{,}}m}
{\iintvargsaux#1}
\NewDocumentCommand{\iintvargsaux}{mm} {#1\mkern1.5mu,\mkern1.5mu#2}

\makeatletter
\newcommand*{\currentname}{\@currentlabelname}
\makeatother

\usepackage[a4paper,hmargin=1in, vmargin=1.4in,footskip=0.25in]{geometry}


%\addtolength{\hoffset}{-1cm}
%\addtolength{\hoffset}{-2.5cm}
%\addtolength{\voffset}{-2.5cm}
\addtolength{\textwidth}{0.2cm}
%\addtolength{\textheight}{2cm}
\setlength{\parskip}{8pt}
\setlength{\parindent}{0.5cm}
\linespread{1.5}

\pagestyle{fancy}
\fancyhf{}
\rhead{TP2 - Sullivan}
\lhead{Seguridad Informática}
\rfoot{\vspace{1cm} \thepage}

\renewcommand*\contentsname{\LARGE Índice}

\begin{document}

\begin{titlepage}
    \begin{center}
        \vfill
        \vfill
            \vspace{0.7cm}
            \noindent\textbf{\Huge Trabajo Práctico 2}\par
            \noindent\textbf{\Huge Seguridad Informática}\par
            \vspace{.5cm}
        \vfill
        \noindent \textbf{\huge Alumna:}\par
        \vspace{.5cm}
        \noindent \textbf{\Large Sullivan, Katherine}\par
 
        \vfill
        \large Universidad Nacional de Rosario \par
        \noindent\large 2023
    \end{center}
\end{titlepage}
\ \par


% Setting the spacing between lines
\setstretch{0}

\section{Buffer Overflow}

\subsection*{Apagando las contramedidas}
Para empezar con el ataque, debemos apagar las contramedidas que se nos indican en el laboratorio. 

\begin{center}
\includegraphics*[scale=0.7, width=1\textwidth]{deshab1.png}
\end{center}

\begin{center}
\includegraphics*[scale=0.8, width=1\textwidth]{deshab2.png}
\end{center}

Además correremos los archivos siguiendo las indicaciones de deshabilitar la \textit{Stack Guard} y
de habilitar la ejecución de stack con \textit{execstack}.

\subsection*{Corriendo el shellcode}
Para familiarizarnos con su funcionamiento, el laboratorio pone como tarea el correr el shellcode, así que eso hacemos

\begin{center}
\includegraphics[scale=0.4]{callshellcode.png}
\end{center}

\subsection{El programa vulnerable}
Se nos provee el código de un programa vulnerable a un ataque de buffer overflow:

\begin{center}
\includegraphics[scale=0.55]{codigostack.png}
\end{center}

y se nos indica 
cambiarle los permisos para que ejecute como root, para que tenga más relevancia 
que la vulnerabilidad del programa sea explotada.

\subsection*{Explotando la vulnerabilidad}

Ahora sí, pasamos a la parte interesante del laboratorio: realizar el ataque. 

Lo que se nos propone hacer es explotar la vulnerabilidad en \textit{bof} de stack.c 
creando un badfile específico en exploit.c:

\begin{center}
\includegraphics[scale=0.55]{codigoexploit.png}
\end{center}


En exploit.c tenemos un programa shellcode. Vamos a querer que este se pueda ejecutar
cuando se ejecute stack.c dejando el código dentro de badfile.

¿Pero dónde lo deberíamos poner en badfile para que se ejecute?

Para lograr esto, lo que queremos hacer es poner en el espacio del return address de \textit{bof},
la dirección donde empieza el codigo de la shellcode. Al código de shellcode lo vamos
poner inmediatamente despues de la return address en la pila.

Queremos entonces que la pila tenga donde va la return address de bof
(que en badfile será donde empieza el array más la distancia a la return address) 
la dirección de donde empieza shellcode (que por donde dijimos que lo ibamos a poner va a ser la 
dirección de la return address más 4 bytes -por el tamaño de la dirección-) 
e inmediatamente después pondremos el código de shellcode. 

Primero, entonces, tenemos que averiguar la distancia a la return address desde el comienzo del array. 
La return address está a 4 bytes de la dirección de ebp. 
Por eso, para averiguarla nos valemos del código assembler de stack
y vemos que el ebp se encuentra a 32 bytes por lo que la distancia a la return address es 36.

\begin{center}
\includegraphics[scale=0.7]{loadeffectiveaddress.png}
\end{center}

Y segundo, tenemos que ver en qué dirección estaría el shellcode. Dijimos que a 4 bytes de la dirección de 
la return address y que la return address está a 4 bytes de ebp. Por lo tanto la dirección donde estaría la
shellcode sería la dirección de ebp más 8 bytes. Si corremos gdb podemos ver la dirección de ebp y entonces
obtener la dirección donde estaría el código de shellcode. 

\begin{center}
\includegraphics[scale=0.5]{findingreturnaddress.png}
\end{center}

Entonces el código modificado quedaría con estos cambios.

\begin{center}
\includegraphics[scale=0.8]{codigomodif.png}
\end{center}

Y vemos que podemos realizar el ataque exitosamente.

\begin{center}
\includegraphics[scale=0.5]{successfulattack.png}
\end{center}

\newpage

\section{PCC}

\subsection*{A. Descripción del programa}

\begin{center}
\includegraphics[scale=0.7]{progPCC.png}
\end{center}

Podemos entender el programa como un bucle entre las 
líneas L1 y L2, el cual se ejecuta hasta que la lista
en $r_2$ queda vacía.

Lo almacenado en $r_2$ comienza siendo la lista argumento $r_0$, lo almacenado en $r_3$, 
una lista vacía y el funcionamiento dentro del bucle 
es el siguiente:

En $r_4$ se guarda la \textit{tail} de la de la lista en $r_2$, mientras que 
se guarda como \textit{next} del primer elemento de la lista $r_2$
a $r_3$, y luego, se pasa esa lista armada por el \textit{head} de $r_2$ y $r_3$ a $r_3$, mientras que
lo guardado en $r_4$ (la \textit{tail}) pasa a $r_2$ y se 
repite el ciclo.

Esto, en resumen, lo que logra es que en $r_3$ se vaya
reconstruyendo lo que había en la lista $r_0$ pero de manera inversa
y esto es lo que devolverá la función. 

\textbf{El programa es una implementación de la función inverse para listas de enteros}

\subsection*{B. Computación de la condición de verificación}

\begin{flalign*}
VC_{13} = true &&\\\nonumber 
\end{flalign*}

\begin{flalign*}
VC_{12} &= VC_{13}[r_3/r_0] &&\\\nonumber 
        &= true &&\\\nonumber 
\end{flalign*}

\begin{flalign*}
VC_{11} &= r_3 :_{m} intlist &&\\\nonumber 
\end{flalign*}

\begin{flalign*}
VC_{10} &= VC_{10+(-6)-1} &&\\\nonumber 
        &= VC_{3} &&\\\nonumber 
        &= r_2 :_m intlist \wedge r_3 :_m intlist \wedge (r_6 = 0) &&\\\nonumber 
\end{flalign*}

\begin{flalign*}
VC_{9} &= VC_{10}[r_4/r_2] &&\\\nonumber 
       &=r_4 :_m intlist \wedge r_3 :_m intlist 
            \wedge (r_6 = 0) &&\\\nonumber 
\end{flalign*}

\begin{flalign*}
VC_{8} &= VC_{9}[r_2/r_3] &&\\\nonumber 
       &= r_4 :_m intlist \wedge r_2 :_m intlist 
\wedge (r_6=0) &&\\\nonumber 
\end{flalign*}


\begin{flalign*}
VC_{7} &= VC_{8}[upd(m, r_2+2, r_3)/m] \wedge writable(r_2+2) &&\\\nonumber 
   &= r_4 :_{upd(m, r_2+2, r_3)} intlist
\wedge r_2 :_{upd(m, r_2+2, r_3)} intlist 
\wedge (r_6=0) \wedge writable(r_2+2) &&\\\nonumber 
\end{flalign*}

\begin{flalign*}
VC_{6} &= VC_{7}[m(r_2+2)/r_4] \wedge readable(r_2+2) &&\\\nonumber 
    &= m(r_2+2) :_{upd(m, r_2+2, r_3)} intlist &&\\\nonumber 
&\wedge r_2 :_{upd(m, r_2+2, r_3)} intlist \wedge r_6=0 \wedge writable(r_2+2) \wedge readable(r_2+2)&&\\\nonumber 
&= r_3 :_m intlist \wedge r_2 :_{upd(m, r_2+2, r_3)} intlist \wedge (r_6=0) \wedge writable(r_2+2) \wedge readable(r_2+2) &&\\\nonumber 
\end{flalign*}

\begin{flalign*}
VC_{5} &= (r_5 = r_6 \rightarrow VC_{5+7-1}) \wedge (r_5 \neq r_6 \rightarrow VC_{6}) &&\\\nonumber 
    &= (r_5 = r_6 \rightarrow VC_{11}) \wedge (r_5 \neq r_6 \rightarrow VC_{6}) &&\\\nonumber 
&= (r_5 = r_6 \rightarrow r_3 :_{m} intlist) &&\\\nonumber 
&\wedge (r_5 \neq r_6 \rightarrow &&\\\nonumber 
& r_3 :_m intlist \wedge r_2 :_{upd(m, r_2+2, r_3)} intlist \wedge (r_6=0) \wedge writable(r_2+2) \wedge readable(r_2+2)) &&\\\nonumber 
\end{flalign*}

\begin{flalign*}
VC_{4} &= VC_{5}[m(r_2+0)/r_5] &&\\\nonumber
    &= (m(r_2) = r_6 \rightarrow r_3 :_{m} intlist) &&\\\nonumber 
    &\wedge (m(r_2) \neq r_6 \rightarrow &&\\\nonumber
    & r_3 :_m intlist \wedge r_2 :_{upd(m, r_2+2, r_3)} intlist \wedge (r_6=0) \wedge writable(r_2+2) \wedge readable(r_2+2)) &&\\\nonumber
\end{flalign*}

\begin{flalign*}
VC_{3} &= r_2 :_m intlist \wedge r_3 :_m intlist \wedge (r_6 = 0) &&\\\nonumber
\end{flalign*}

\begin{flalign*}
VC_{2} &= VC_{3}[r_1/r_3] &&\\\nonumber 
       &= r_2 :_m intlist \wedge r_1 :_m intlist \wedge (r_6 = 0) &&\\\nonumber
\end{flalign*}

\begin{flalign*}
VC_{1} &= VC_{2}[r_0/r_2] &&\\\nonumber 
&= r_0 :_m intlist \wedge r_1 :_m intlist \wedge (r_6 = 0) &&\\\nonumber
\end{flalign*}

Ahora, la condición de verificación se calcula de la siguiente manera: 

$$ VC(inverse, Inv) = \forall r_i : \bigwedge_{i \in Inv} Inv_i \implies VC_{i+1}$$

donde $Inv$ es el conjunto de los números de línea donde hay una instrucción
\textit{Precondition} o \textbf{INV} e $Inv_i$ es el predicado 
correspondiente a la instrucion de la línea $i$.

Entonces, calculemos: 

$Inv = \{0, 3, 11\}$

Por lo que tendremos 3 implicancias en conjunción:

\begin{enumerate}
    \item $Inv_0 \implies VC_1$
    \item $Inv_3 \implies VC_4$
    \item $Inv_{11} \implies VC_{12}$
\end{enumerate}

Es decir, tendremos la conjunción de:

\begin{enumerate}
\item $(r_0 :_m intlist \wedge r_1 :_m intlist \wedge tag(r_1) = 0 \wedge r_6 = 0) \implies 
r_0 :_m intlist \wedge r_1 :_m intlist \wedge (r_6 = 0)$

\item \begin{flalign*}
    (r_2 :_m intlist &\wedge r_3 :_m intlist \wedge (r_6 = 0)) \implies  &&\\\nonumber
    &(m(r_2) = r_6 \rightarrow r_3 :_{m} intlist) &&\\\nonumber 
    &\wedge (m(r_2) \neq r_6 \rightarrow r_3 :_m intlist &&\\\nonumber
    &\wedge r_2 :_{upd(m, r_2+2, r_3)} intlist \wedge (r_6=0) \wedge writable(r_2+2) \wedge readable(r_2+2))
\end{flalign*}

\item $r_3 :_{m} intlist \implies true$

\end{enumerate}

Vemos que la conjunción vale al valer cada parte de ella:

\begin{enumerate}
\item Tenemos lo mismo en ambos lados de la implicancia por lo que vale. 
\item Tenemos una conjunción como secuente y vemos que vale cada parte: 
la primera trivialmente por hipótesis y la segunda, sabiendo que
$upd(m, r_2+2, r_3)$ no cambia a $r_2$, por hipotésis vale $r_3 :_m intlist
\wedge r_2 :_{upd(m, r_2+2, r_3)} intlist \wedge (r_6=0)$, y porque $r_2+2$
es el puntero al próximo elemento de la lista $r_2$ vale que es $readable$ y $writable$,
con lo que vale la conjunción.
\item Tenemos algo de la forma $p \implies true$ que siempre vale.
\end{enumerate}

\subsection*{C. Modificación para listas ordenadas}

Para poder condicionar que las listas est\'en ordenadas se podr\'ia imponer una condici\'on 
$ordered\_intlist(v, m)$ formada con las f\'ormulas at\'omicas y las f\'ormulas del lenguaje presentadas en
la secci\'on 3.1. 

Definamos $ordered\_intlist(v, m)$ como sigue:

\begin{flalign*}
ordered\_intlist(v, m) &= v :_m intlist &&\\\nonumber
&\wedge (m(v) = 0 &&\\\nonumber
&\vee (m(v) = 1 \wedge ordered\_intlist(m(v+2), m) \wedge m(v+1) \leq m(m(v+2)+1))) &&\\\nonumber
\end{flalign*}

Con esta definición los únicos cambios en el código que habría que hacer
para que solo se tomen lisats ordenadas de input serían que
en la precondición tendríamos:

$ordered\_intlist(r_0, m) \wedge ordered\_intlist(r_1, m) \wedge tag(r_1) = 0 \wedge r_6 = 0$

y en la instrucción 3 tendríamos:

\textbf{INV} $ordered\_intlist(r_2, m) \wedge r_3 :_m intlist \wedge r_6 = 0$

\newpage

\section{Criptografía}

\subsection*{Ejercicio 2}
Se nos solicita responder cómo funciona la desencriptación para DES con dos claves y 
qué vulnerabilidades podría tener este esquema.

La idea de utilizar más de una clave para aumentar la longitud efectiva de la clave en DES es la idea 
que se utiliza en el conocido Triple DES. Aunque este enfoque difiere
al tener solo 2 claves, surge también de la vulnerabilidad frente a ataques por fuerza bruta 
que comienza a generar el solo contar con claves de 56 bits.

El mensaje se cifraría y luego se descifraría de la siguiente manera:

\begin{itemize}
\item Cifrado:

\begin{itemize}
\item Paso 1: Se cifra el mensaje original con la primera clave (k1).
\item Paso 2: Se cifra el resultado del paso 1 con la segunda clave (k2).
\end{itemize}

\item Descifrado:

\begin{itemize}
\item Paso 1: Se descifra el mensaje con la segunda clave (k2).
\item Paso 2: Se descifra el resultado del paso 2 con la primera clave (k1).
\end{itemize}

\end{itemize}

La principal ventaja de este enfoque es que ofrece mayor seguridad debido a su mayor longitud de clave. 
Sin embargo, si un atacante tiene mucha memoria disponible, podría intentar realizar ataques de búsqueda exhaustiva 
almacenando todos los resultados intermedios de las claves intermedias en la fase de cifrado y lograr producir un ataque.

Si bien este nuevo enfoque ofrece cierta resistencia a este tipo de ataques (en especial respecto al DES original)
debido a su longitud de clave efectiva, 
aún así, la seguridad depende de la complejidad y aleatoriedad de las claves utilizadas (por ejemplo, si
se tomasen dos claves iguales sería equivalente a DES).

\subsection*{Ejercicio 6}

Se nos solicita responder cuál es la diferencia entre los modos de operación ECB y CBC y 
cuál recomendaríamos para
encriptar el contenido de una imagen en forma de mapa de bits. 

Los modos de operación ECB (Electronic Codebook) y CBC (Cipher Block Chaining) son dos enfoques 
diferentes para cifrar datos mediante DES.

Mientras ECB divide el mensaje en bloques de datos fijos y cifra cada bloque de forma independiente utilizando la misma clave, 
haciendo que los bloques de cifrado sean independiente entre sí, el CBC 
introduce un vector de inicialización (IV) que se combina (con XOR) con el primer bloque de datos antes de cifrar y
el bloque cifrado resultante se combina con el siguiente bloque de datos antes de cifrarlo, haciendo que 
cada bloque cifrado dependa del bloque anterior.

Esto produce que si bien ECB es más paralelizable, este puede revelar patrones en el texto original 
al existir la posibilidad de haber bloques idénticos que se cifren de la misma manera. 

Por lo tanto la recomendación para encriptar una imagen en forma de mapa de bits sería el modo de 
operación CBC, puesto que en imágenes es bastante probable el encontrarse con bloques repetidos.


\end{document}